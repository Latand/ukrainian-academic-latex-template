% ===============================================================
%  Ukrainian Academic Document Template
%  Compile with: lualatex + biber
% ===============================================================

\documentclass[
  12pt,
  oneside,
  headsepline,
  onehalfspacing,
  chapterinoneline
]{UkrainianAcademic}

% === CONFIGURE YOUR DOCUMENT HERE ===
\thesistitle{Назва вашого документа}
\author{Прізвище Ім'я По батькові}
\supervisor{д.т.н., проф. Прізвище І.П.}
\university{Назва університету}
\faculty{Назва факультету}
\department{Назва кафедри}
\specialty{000 Назва спеціальності}
\keywords{ключове слово 1, ключове слово 2, ключове слово 3}
\keywordseng{keyword 1, keyword 2, keyword 3}
% =====================================

% === LANGUAGE SUPPORT ===
\usepackage{polyglossia}
\setdefaultlanguage{ukrainian}
\setotherlanguage{english}
\newfontfamily\cyrillicfonttt{Times New Roman}[Script=Cyrillic]

% === TYPOGRAPHY ===
\usepackage{microtype}
\emergencystretch=3em
\setlength{\parindent}{1.25cm}

% === MATH AND TABLES ===
\usepackage{amsmath}
\usepackage{mathtools}
\usepackage{array}

% === LISTS ===
\usepackage{enumitem}
\setlist{nosep, leftmargin=1.25cm}

% === CHAPTER FORMATTING ===
\usepackage{titlesec}
\titleformat{\chapter}{\Large\bfseries\centering}
  {\MakeUppercase{РОЗДІЛ~\thechapter.}}{1em}{\MakeUppercase}
\titlespacing*{\chapter}{0pt}{0pt}{0.5\baselineskip}
\titleformat{\section}{\large\bfseries}{\thesection}{1em}{}
\titlespacing*{\section}{0pt}{0.5\baselineskip}{0.5\baselineskip}

% === TEXT QUALITY ===
\tolerance=1000
\hbadness=10000
\vbadness=10000
\hyphenpenalty=50
\exhyphenpenalty=50
\doublehyphendemerits=10000
\finalhyphendemerits=5000

% === BIBLIOGRAPHY ===
\usepackage[
  backend=biber,
  style=numeric,
  sorting=none,
  maxbibnames=10,
  giveninits=true
]{biblatex}
\addbibresource{refs.bib}

% Ukrainian bibliography strings
\DeclareFieldFormat{url}{\bibstring{url}\addcolon\space\url{#1}}
\DeclareFieldFormat{doi}{DOI:\space\url{https://doi.org/#1}}
\DefineBibliographyStrings{ukrainian}{
  url={URL},
  and={та},
}

% === PDF METADATA ===
\usepackage{hyperref}
\hypersetup{
  pdftitle={\ttitle},
  pdfauthor={\authorname},
  pdfsubject={},
  pdfkeywords={\keywordnames},
  hidelinks
}

% ===============================================================
%  DOCUMENT CONTENT
% ===============================================================

\begin{document}

% --- Title page ---
% ===============================================================
%  Title Page Template
%  Edit the metadata in main.tex, not here
% ===============================================================

\begin{titlepage}
\begin{center}

% --- Ministry header (optional, remove if not needed) ---
\textsc{Міністерство освіти і науки України}

\vspace{0.3cm}

% --- University name ---
\textsc{\univname}

\vspace{0.3cm}

% --- Faculty and department ---
\facname\\
Кафедра \deptname

\vspace{1.5cm}

% --- Document type ---
{\large\bfseries РЕФЕРАТ}

\vspace{0.3cm}

% --- Course name (optional) ---
з дисципліни <<Назва дисципліни>>

\vspace{0.5cm}

% --- Document title ---
на тему:

\vspace{0.3cm}

{\Large\bfseries <<\ttitle>>}

\vfill

% --- Author and supervisor ---
\begin{flushright}
\begin{minipage}{0.5\textwidth}
\begin{flushleft}
Виконав:\\
\authorname

\vspace{0.5cm}

Науковий керівник:\\
\supname
\end{flushleft}
\end{minipage}
\end{flushright}

\vfill

% --- City and year ---
Одеса --- \the\year

\end{center}
\end{titlepage}


% --- Table of contents ---
\tableofcontents

% --- Main content ---
% ===============================================================
%  Introduction (Вступ)
%  This is an unnumbered chapter
% ===============================================================

\chapter*{ВСТУП}
\addcontentsline{toc}{chapter}{ВСТУП}

Вступ є важливою частиною наукової роботи, що обґрунтовує актуальність обраної теми та формулює мету дослідження.

Структура вступу зазвичай включає:

\textbf{Актуальність теми.} Обґрунтування важливості дослідження в сучасних умовах. Чому ця тема є важливою? Які проблеми вона вирішує?

\textbf{Мета роботи.} Чітке формулювання того, чого планується досягти в результаті дослідження.

\textbf{Завдання дослідження.} Конкретні кроки для досягнення мети:
\begin{itemize}
    \item проаналізувати теоретичні засади;
    \item дослідити практичні аспекти;
    \item розробити рекомендації.
\end{itemize}

\textbf{Об'єкт дослідження.} Загальна сфера, в межах якої проводиться дослідження.

\textbf{Предмет дослідження.} Конкретний аспект об'єкта, що досліджується.

\textbf{Методи дослідження.} Перелік наукових методів, використаних у роботі.

\textbf{Структура роботи.} Короткий опис розділів документа.

% ===============================================================
%  Chapter 1 Template
%  Numbered chapter with sections
% ===============================================================

\chapter{НАЗВА ПЕРШОГО РОЗДІЛУ}

Це шаблон розділу з секціями та підсекціями. Кожен розділ зазвичай починається з короткого вступного абзацу, що окреслює його зміст.

\section{Перша секція}

Текст першої секції. Для посилання на джерела використовуйте команду \verb|\cite{key}|, наприклад: за даними дослідження \cite{example_book}.

Приклад формули:
\begin{equation}
    E = mc^2
\end{equation}

\section{Друга секція}

Текст другої секції з прикладом таблиці.

\begin{table}[htbp]
    \centering
    \caption{Приклад таблиці}
    \begin{tabular}{lcc}
        \toprule
        Параметр & Значення 1 & Значення 2 \\
        \midrule
        A & 10 & 20 \\
        B & 15 & 25 \\
        C & 12 & 18 \\
        \bottomrule
    \end{tabular}
    \label{tab:example}
\end{table}

\subsection{Підсекція}

Текст підсекції. Посилання на таблицю: див. табл.~\ref{tab:example}.

\section{Третя секція}

Приклад рисунка (розкоментуйте та додайте файл в images/):

% \begin{figure}[htbp]
%     \centering
%     \includegraphics[width=0.7\textwidth]{example-image}
%     \caption{Приклад рисунка}
%     \label{fig:example}
% \end{figure}

% Посилання на рисунок: див. рис.~\ref{fig:example}.

Завершальна секція розділу з узагальненням основних положень.

% Add more chapters here:
% \input{chapters/chapter2}
% \input{chapters/chapter3}
% ===============================================================
%  Conclusions (Висновки)
%  Unnumbered chapter
% ===============================================================

\chapter*{ВИСНОВКИ}
\addcontentsline{toc}{chapter}{ВИСНОВКИ}

Висновки є заключною частиною роботи, де підсумовуються основні результати дослідження.

Рекомендована структура висновків:

По-перше, стислий виклад першого ключового результату дослідження.

По-друге, формулювання другого важливого висновку.

По-третє, узагальнення практичного значення отриманих результатів.

Нарешті, окреслення перспектив подальших досліджень у даному напрямку.

Висновки мають логічно випливати зі змісту роботи та відповідати поставленим у вступі завданням. Кожен пункт висновків має бути конкретним та обґрунтованим.


% --- Bibliography ---
\printbibliography[heading=bibintoc, title={СПИСОК ВИКОРИСТАНИХ ДЖЕРЕЛ}]

\end{document}
