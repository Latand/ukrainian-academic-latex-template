% ===============================================================
%  Introduction (Вступ)
%  This is an unnumbered chapter
% ===============================================================

\chapter*{ВСТУП}
\addcontentsline{toc}{chapter}{ВСТУП}

Вступ є важливою частиною наукової роботи, що обґрунтовує актуальність обраної теми та формулює мету дослідження.

Структура вступу зазвичай включає:

\textbf{Актуальність теми.} Обґрунтування важливості дослідження в сучасних умовах. Чому ця тема є важливою? Які проблеми вона вирішує?

\textbf{Мета роботи.} Чітке формулювання того, чого планується досягти в результаті дослідження.

\textbf{Завдання дослідження.} Конкретні кроки для досягнення мети:
\begin{itemize}
    \item проаналізувати теоретичні засади;
    \item дослідити практичні аспекти;
    \item розробити рекомендації.
\end{itemize}

\textbf{Об'єкт дослідження.} Загальна сфера, в межах якої проводиться дослідження.

\textbf{Предмет дослідження.} Конкретний аспект об'єкта, що досліджується.

\textbf{Методи дослідження.} Перелік наукових методів, використаних у роботі.

\textbf{Структура роботи.} Короткий опис розділів документа.
