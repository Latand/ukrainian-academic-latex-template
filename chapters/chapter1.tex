% ===============================================================
%  Chapter 1 Template
%  Numbered chapter with sections
% ===============================================================

\chapter{НАЗВА ПЕРШОГО РОЗДІЛУ}

Це шаблон розділу з секціями та підсекціями. Кожен розділ зазвичай починається з короткого вступного абзацу, що окреслює його зміст.

\section{Перша секція}

Текст першої секції. Для посилання на джерела використовуйте команду \verb|\cite{key}|, наприклад: за даними дослідження \cite{example_book}.

Приклад формули:
\begin{equation}
    E = mc^2
\end{equation}

\section{Друга секція}

Текст другої секції з прикладом таблиці.

\begin{table}[htbp]
    \centering
    \caption{Приклад таблиці}
    \begin{tabular}{lcc}
        \toprule
        Параметр & Значення 1 & Значення 2 \\
        \midrule
        A & 10 & 20 \\
        B & 15 & 25 \\
        C & 12 & 18 \\
        \bottomrule
    \end{tabular}
    \label{tab:example}
\end{table}

\subsection{Підсекція}

Текст підсекції. Посилання на таблицю: див. табл.~\ref{tab:example}.

\section{Третя секція}

Приклад рисунка (розкоментуйте та додайте файл в images/):

% \begin{figure}[htbp]
%     \centering
%     \includegraphics[width=0.7\textwidth]{example-image}
%     \caption{Приклад рисунка}
%     \label{fig:example}
% \end{figure}

% Посилання на рисунок: див. рис.~\ref{fig:example}.

Завершальна секція розділу з узагальненням основних положень.
